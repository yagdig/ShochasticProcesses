% LaTeX source for book ``代数学方法'' in Chinese
% Copyright 2018  李文威 (Wen-Wei Li).
% Permission is granted to copy, distribute and/or modify this
% document under the terms of the Creative Commons
% Attribution 4.0 International (CC BY 4.0)
% http://creativecommons.org/licenses/by/4.0/

% To be included
\chapter{Brown运动}

\begin{compactitem}
	\item Brown运动的定义.
	\item 正态增量、独立增量和路径连续性.
	\item Gauss过程的特点.
\end{compactitem}

\begin{definition}\label{def:BrownMotion}\index{brownyundong@Brown运动(Brown Motion)}
	随机过程\(\{X(t),t \geqslant 0 \}\)如果满足:
	\begin{enumerate}[\bfseries (1)]
		\item \(X(0)=0\);
		\item \(\{X(t),t\geqslant 0\}\);
		\item 对每个\(t>0,X(t)\)服从正态分布\(N(0,\sigma^2t)\)。
	\end{enumerate}
	则称\(X(t),t \geqslant 0\)为\textbf{Brown运动},也称为\emph{Wiener过程},记为\(\{B(t),t\geqslant0\}\)或\(\{W(t),t\geqslant0\}\)。
\end{definition}

Brown运动\(\{B(t),t,\geqslant 0\}\)具有以下性质:
\begin{itemize}[\bfseries (1)]
	\item (正态增量过程)\(\forall 0\leqslant s <t,B(t)-B(s)\sim N(0,t-s)\),即\(\{B(t)-B(s)\}\)服从均值为0,方差为\(t-s\)的正态分布。当\(s=0\)时,\(B(t)-B(0)\sim N(0,t)\)。
	\item (独立增量)\(\forall 0\leqslant s <t,B(t)-B(s)\)独立于过程过去的状态\(B(u),0\leqslant u<s\)。
	\item (路径的连续性)\(\{B(t)(t\geqslant0)\}\)是t的连续函数。
\end{itemize}

不失一般性,假设\(B(0)=0\)。

\begin{definition}\label{def:QuadraticVariation}\index{ercibiancha@二次变差(Quadratic Variation)}
	Brown运动的\emph{二次变差}\([B,B](t)\)定义为党\(\{t_i^n\}_{i=0}^n\)取遍\([0,t]\)的分割,且其模
	\(\delta_n=\underset{0\leqslant i\leqslant n-1}{\max}\{t_{i+1}^{n}-t_{i}^n\}\to 0\)时,依概率收敛意义下的极限
	\begin{align*}
		[B,B](t)=[B,B]([0,t])=\underset{\delta_n \to 0}{\lim}\sum_{i=0}^{n-1}
		\lvert B(t_{i+1}^n)-B(t_i^n) \rvert^2
	\end{align*}
\end{definition}

类似的,一次变差的定义为\(\underset{\delta_n \to 0}{\lim}\sum_{i=0}^{n-1}\lvert B(t_{i+1}^n)-B(t_i^n) \rvert\)。

从时刻0到时刻t对Brown运动的一次观察称为Brown运动在区间[0,T]上的一个路径或一个实现。Brown运动的几乎所有样本路径
\(B(t)(0\leqslant t\leqslant T)\)都具有下述性质:
\begin{enumerate}[\bfseries (1)]
	\item 是\(t\)的连续函数;
	\item 在任意区间(无论区间多么小)上都不是单调的;
	\item 在任意点都不是可微的;
	\item 在任意区间(无论区间多么小)上都是无限变差的(一次变差);
	\item 对任意\(t\),在区间\([0,t]\)上的二次变差等于\(t\)。
\end{enumerate}

对任意过程,如果一次变差是有限值,那么二次变差一定是0。对于Brown运动,二次变差是有限值,一次变差是无穷大。

\begin{theorem}[多元正态分布的性质]\label{prop:MultiNormDist}
	设\(X\sim \mathcal{N}(\mu_1,\sigma^2_1),Y\sim \mathcal{N}(\mu_2,\sigma_2^2)\)是相互独立的,
	则\((X+Y)\sim \mathcal{N}(\boldsymbol{\mu},\boldsymbol{\Sigma})\)。其中均值
	\(\boldsymbol{\mu }=(\mu_1,\mu_1+mu_2)'\),协方差矩阵
	\(\boldsymbol{\Sigma}=\begin{pmatrix}
		\sigma_1^2 & \sigma_1^2            \\
		\sigma_1^2 & \sigma_1^2+\sigma_2^2
	\end{pmatrix}\)。
\end{theorem}

\begin{theorem}[Brown运动是Gauss过程]\label{prop:GaussProcess}
	Brown运动是均值函数为\(m(t)=0\)、协方差函数为\(\gamma (s,t)=\min\{t,s\}\)的Gauss过程。
\end{theorem}

\newpage
\begin{Exercises}
	\item 设\(\{B(t),t\geqslant0\}\)是标准Brown运动,计算\(P\{B(2)\leqslant0\}\)和\(P\{B(t),\leqslant0,t=0,1,2\}\)。
	\vspace{30em}
	\item 设\(\{B(t)\}\)是Brown运动,求\(B(1)+B(2)+B(3)+B(4)\)的分布。
	\newpage
	\item 求\(B(\frac{1}{4})+B(\frac{1}{2})+B(\frac{3}{4})+B(1)\)的分布.
	\vspace{30em}
	\item 求概率\(P\{\int_{0}^{1} B(t)dt>\frac{2}{\sqrt{3}}\}\)
\end{Exercises}