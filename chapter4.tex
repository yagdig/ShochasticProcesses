% LaTeX source for book ``代数学方法'' in Chinese
% Copyright 2018  李文威 (Wen-Wei Li).
% Permission is granted to copy, distribute and/or modify this
% document under the terms of the Creative Commons
% Attribution 4.0 International (CC BY 4.0)
% http://creativecommons.org/licenses/by/4.0/

% To be included
\chapter{鞅}

\begin{compactitem}
	\item 鞅的定义.
	\item 停时定理.
\end{compactitem}

\begin{definition}\label{def:martingale}\index{Yang@鞅 (Martingale)}
	如果\(\{\mathscr{F}_n,n\geqslant0\}\)是一个\(\mathscr{F}\)中的单调递增的子\(\sigma\)代数流。随机过程\(\{X_n,n\geqslant0\}\)称为关于\(\{\mathscr{F}_n,n\geqslant0\}\)的\emph{鞅},如果\(\{X_n\}\)是\(\{\mathscr{F}_n\}\)适应的,\(E(\lvert X_n\rvert)<\infty\),并且\(\forall n\geqslant0\),有
	\begin{align*}
		E(X_{n+1}|\mathscr{F}_n)=X_n
	\end{align*}
	适应列\(\{X_n,\mathscr{F}_n,n\geqslant0\}\)称为\emph{下鞅},如果\(\forall n \geqslant 0\),有
	\begin{align*}
		E(X_n^+)<\infty, \quad E(X_{n+1}|\mathscr{F}_n)\geqslant X_n
	\end{align*}
	\emph{上鞅}可类似定义。
\end{definition}

\(\sigma\)代数的概念是在提出概率空间时引入的,它是样本空间的某些子集组成的集合组。它表示这些子集可以被我们所度量,比如我们关心事件A发生的概率,我们也会去关系事件A不发生的概率,如果还有一个事件B,我们也会关心事件A和事件B同时发生的概率,等等。\(\sigma\)代数就是所有这些事件的组合的集合,这些事件的组合应当也是样本空间的子集。严格意义上,\(\sigma\)代数要满足补集运算和可数个并集运算的封闭性。

\(\sigma\)域流表示随时间演化的信息流,随机变量随着时间的增加产生更多的信息。\(\sigma\)域流需要满足信息随时间只增不减,以及\(\sigma\)代数的可测性。

不懂这些数学概念并不影响做题,因为通常情况下,\(\sigma\)代数流的要求都可以满足。但是需要知道,\textbf{鞅}是定义在一种规则(或信息)下的,在这个规则(或信息)下,随机过程的条件期望满足一些性质,离开了这个规则(或信息),鞅的定义便不存在。在鞅的定义中,条件期望所满足的条件,是用来描述这种规则(或信息)的。

\begin{theorem}[停时定理]\label{prop:Downtime}
	设\(\{X_n,n\geqslant 0\}\)是一个随机变量序列,称随机函数T是关于\(\{X_n,n\geqslant 0\}\)的\emph{停时},如果T在\(\mathbb{Z^+}\)中取值,且对每个\(n\geqslant 0,\{T=n\}\in \sigma(X_0,X_1,\cdots,X_n)\)。

	如果\(\{X_n,n\geqslant0\}\)是关于\(\{\mathscr{F}_n=\sigma(X_0,X_1,\cdots,X_n)\}\)的鞅,T是停时且满足
	\begin{enumerate}[\bfseries (1)]
		\item \(P\{T<\infty\}\),
		\item \(E(|M_T|)<+\infty\),
		\item \(\lim_{n\to \infty}E(|M_n|I_{\{T>n\}})=0\)
	\end{enumerate}
	则有
	\begin{align*}
		E(M_T)=E(M_0)
	\end{align*}
\end{theorem}

由定义可以发现\(\{T=n\}\)或\(\{T\neq n\}\)都应当由\(n\)时刻及其之前的信息完全确定,而不需要也无法借助将来的情况。借助赌博的例子,赌博者何时决定停止赌博只能以他已经赌过的结果为依据,而不能说,如果下一次要输,我现在就停止赌博,这时对停止时刻T的第一个要求:它必须是一个停时。

\begin{Exercises}
	\item 设\(Y_0,Y_1,Y_2,\cdots\)是一系列独立随机变量且\(E\lvert Y_n\rvert<+\infty,EY_n=0\)。令\(X_n=\sum_{i=0}^{n}Y_i,n=0,1,2,\cdots,\mathscr{F}_n=\sigma\{Y_i,i=0,1,2,\cdots,n\}\),则\(\{X_n,\mathscr{F}_n\}\)是一个鞅。
	\newpage
	\item 设\(Y_0,Y_1,Y_2,\ldots ,Y_n\)如上题假定,另外设\(EY^2_n=\sigma^2,n=1,2,\ldots,\)令\(Z_0,Z_n=\left(\sum_{i=1}^{n}Y_i\right)^2-n\sigma^2\),那么\(\{Z_n,\mathscr{F}_n\}\)是一个鞅。
	\vspace{30em}
	\item 令\(X_0,X_1,\cdots\)表示分支过程各代的个体数,\(X_0=1\),任意一个个体生育后代的分布有均值\(\mu\)。证明\(\{M_n=\mu^{-n}X_n\}\)是一个关于\(X_0,X_1,\cdots\)的鞅。
	\newpage
	\item 考虑一个在整数上的随机游动模型,设向右移动的概率\(p<\frac{1}{2}\),向左移动的概率为\(1-p,S_n\)表示时刻n所处的位置,假定\(S_0=a,0<a<N\)。
	\begin{enumerate}[\bfseries (1)]
		\item 证明:\(\{M_n=\left(\frac{1-p}{p}\right)^{S_n}\}\)是鞅;
		\item 令T表示随机游动第一次到达0或N的时刻,即
		      \begin{align*}
			      T=\min\{n:S_n=0orN\}
		      \end{align*}
		      利用鞅停时定理,求出\(P\{S_T=0\}\)。
	\end{enumerate}
\end{Exercises}