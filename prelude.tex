% LaTeX source for book ``代数学方法'' in Chinese
% Copyright 2018  李文威 (Wen-Wei Li).
% Permission is granted to copy, distribute and/or modify this
% document under the terms of the Creative Commons
% Attribution 4.0 International (CC BY 4.0)
% http://creativecommons.org/licenses/by/4.0/

% To be included
\chapter*{导言}	% 文档类会自动将之加入目录并设置天眉

首先分享几条清华大学张颢老师的几条格言:
\begin{itemize}
	\item No Reading, No Learning.参考多本教程或教材可以让你理解更深刻,但\textbf{一定}要找一个教程深入学习,其他作为参考和辅助。大量的阅读是学习的重要内容。
	\item No Writing, No Reading(Effective Reading).把读到的每个符号都在纸上写一遍。
	\item No Data, No Truth.所有的模型都是人造的,只有数据是上帝给的,数据才是对真理最无私的反映。
	\item No Analytic, No Understanding.一件事在解析上算不清楚就不叫真正理解。
	\item No Programing, No Cognition.理解停留在上层,而认知需要落地。
\end{itemize}

根据费曼学习法,将学到的知识讲述出来可以加深自己的理解,同时也是一个重新整理思路的过程,这是整理这篇文档的初衷。文档中的定义、定理和证明等内容为求准确,完全与《应用随机过程(第6版)》内容一致;文字描述少部分参照教材中描述,大多数是个人的理解,所以难免有偏差;习题选自教材例题和习题(龙汉:我建议大家关注),留出了很多空白,需要动手去完成。

本文档\LaTeX 模板来自北京大学数学科学学院李文威老师《代数学方法》(\href{https://github.com/wenweili/AlJabr-1}{GitHub链接}),节约了很多精力,在此作出感谢。 \nopagebreak

\vspace{1em}
\begin{flushright}\begin{minipage}{0.3 \textwidth}
		\begin{tabular}{c}
			{\kaishu 杨鼎} \\
			2024 年 5 月于北斗楼309
		\end{tabular}
	\end{minipage}\end{flushright}
\vspace{1em}

